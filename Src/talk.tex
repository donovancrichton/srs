\documentclass{beamer}
\usepackage[utf8]{inputenc}

\usepackage{minted}
\usepackage{pgf}
\usepackage{tikz}
\usepackage{upquote}
\usepackage{hyperref}
\usepackage{graphicx}
\usetikzlibrary{arrows,automata,positioning}

\setbeamertemplate{footline}[frame number]
\setbeamertemplate{navigation symbols}{}

\title{Verified Time Balancing of Securit Protocols}
\author{Donovan Crichton}
\date{January 2019}

\begin{document}
 
\frame{\titlepage}

\begin{frame}[fragile]
  \frametitle{The Problem}
  \begin{itemize}
    \item Can we formally prove that an implementation of
            a network security protocol is immune to 
                  timing side-channel attacks?
    \item In other words, can we show that a model of the protocol,
            along with its implementation is unable to leak
                  any timing information to an observer.
    \item Some Example?
  \end{itemize}
\end{frame}

\begin{frame}[fragile]
  \frametitle{Rationale}
  \begin{itemize}
    \item Some departments in the ASD spend a large amount of time
            manually verifying cryptographic routines in vendor
            code.
    \item Requires large amounts of time and expertise, resulting
            in a slow verfication process.
    \item Formal methods may allow progress towards the 
            automation of these verification processes.
  \end{itemize}
\end{frame}

\begin{frame}[fragile]
  \frametitle{Formal Methods}
    \begin{itemize}
      \item A mathematical approach to the 
              development, specification and verification of 
                    software.
      \item Uses "theorem proving assistants", software that
              formally verifies against a specification.
      \item We can implement a proof of concept network security 
              protocol inside a theorem prover (namely Idris).
    \end{itemize}
\end{frame}

\begin{frame}[fragile]
  \frametitle{How do proof assistants work?}
    \begin{itemize}
      \item Rely on on a relationship between proofs in mathematics
              and computer programs known as the Curry-Howard 
                    correspondance.
      \item The proof system of intuitionistic 
              natural deduction can be directly interpreted 
                    as a model of computation (lambda calculus).
    \end{itemize}
\end{frame}

\begin{frame}[fragile]
  \frametitle{A quick Idris example.}
  \begin{block}{A simple function in Idris}
  \begin{minted}{haskell}
f : Nat -> Nat
f x = x + 2
  \end{minted}
  \end{block}
  \begin{block}{The same function in familiar notation.}
$f : \mathbb{N} \mapsto \mathbb{N}$ \\
$f(x) = x + 2$
  \end{block}
\end{frame}


\begin{frame}[fragile]
 \frametitle{Logic in the Idris language.}
    \begin{table}[h!]
    \begin{tabular}{c|c|c|c}
    \textbf{Logic Term} & \textbf{Logic Symbol} & 
            \textbf{Idris Symbol} & \textbf{Idris Term} \\
    \hline
      Implication & p $\Rightarrow$ q & p \mintinline{haskell}{->} q
      & Arrow \\
      Conjunction & p $\land$ q & \mintinline{haskell}{(p, q)}
      & Pair (Product) \\ 
      Disjunction & p $\lor$ q & \mintinline{haskell}{Either p q}
      & Enum (Sum)\\
      Negation & $\lnot$ p & \mintinline{haskell}{p -> Void} &
      Void Type \\ 
      IFF/Eq & p $\equiv$ q, p $\iff$ q & \mintinline{haskell}{(p -> q, q -> p)}  
      & Pair Arrows \\
      Universal & $\forall$ x. P x &
      \mintinline{haskell}{p -> Type} & $\Pi$ Type \\
      Existential & $\exists$ x. P x
      & \mintinline{haskell}{(x ** P x)} & $\Sigma$ Type \\
      \hline
       & & \mintinline{haskell}{p = q} & Type Equality
    \end{tabular}
  \end{table}
\end{frame}


\begin{frame}[fragile]
  \frametitle{Outline of the problem.}
  \begin{itemize}
  \item It's much simpler to prove properties for a small
        language than a large one.
  \item We can then design a minimal programming language
        that is just powerful enough to express our example 
        protocol, while proving that this language has
        the properties we desire.
  \item We could then design a provably correct translation
        from this language into one more amenable to 
        production (C, Assembly, etc).
\end{frame}

\begin{frame}[fragile]
  \frametitle{Prg : A minimal core language}
  \begin{item} Every expression in this language is paramterised
          over its supposed running time. 
  \begin{item} The language can be elaborated into a language nicer
          to use.
  \begin{item} The language can also be compiled or translated
          into a language more amenable for real-life performance.
\end{frame}

\begin{frame}[fragile]
  \frametitle{A diagram of Prg syntax.}
   How should I represent this?
\end{frame}

\begin{frame}[fragile]

\begin{frame}[fragile]
  \frametitle{Challenges and Outcomes}
  \begin{itemize}
    \item Still in the process of confirming whether Prf is 
            expressive enough for a cryptographic key exchange
    \item Are there other kinds of proofs we might be interested in
            on this language to ensure no additional leakage of
                  information?
    \item For future work, the implementation of the compilation
            or translation step in the thorem prover.
\end{frame}


%\begin{frame}[fragile]

%\end{frame}



\end{document}
